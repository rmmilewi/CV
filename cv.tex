%-----------------------------------------------------------------------------------------------------------------------------------------------%
%	The MIT License (MIT)
%
%	Copyright (c) 2021 Jitin Nair
%
%	Permission is hereby granted, free of charge, to any person obtaining a copy
%	of this software and associated documentation files (the "Software"), to deal
%	in the Software without restriction, including without limitation the rights
%	to use, copy, modify, merge, publish, distribute, sublicense, and/or sell
%	copies of the Software, and to permit persons to whom the Software is
%	furnished to do so, subject to the following conditions:
%	
%	THE SOFTWARE IS PROVIDED "AS IS", WITHOUT WARRANTY OF ANY KIND, EXPRESS OR
%	IMPLIED, INCLUDING BUT NOT LIMITED TO THE WARRANTIES OF MERCHANTABILITY,
%	FITNESS FOR A PARTICULAR PURPOSE AND NONINFRINGEMENT. IN NO EVENT SHALL THE
%	AUTHORS OR COPYRIGHT HOLDERS BE LIABLE FOR ANY CLAIM, DAMAGES OR OTHER
%	LIABILITY, WHETHER IN AN ACTION OF CONTRACT, TORT OR OTHERWISE, ARISING FROM,
%	OUT OF OR IN CONNECTION WITH THE SOFTWARE OR THE USE OR OTHER DEALINGS IN
%	THE SOFTWARE.
%	
%
%-----------------------------------------------------------------------------------------------------------------------------------------------%

%----------------------------------------------------------------------------------------
%	DOCUMENT DEFINITION
%----------------------------------------------------------------------------------------

% article class because we want to fully customize the page and not use a cv template
\documentclass[a4paper,12pt]{article}

%----------------------------------------------------------------------------------------
%	FONT
%----------------------------------------------------------------------------------------

% % fontspec allows you to use TTF/OTF fonts directly
% \usepackage{fontspec}
% \defaultfontfeatures{Ligatures=TeX}

% % modified for ShareLaTeX use
% \setmainfont[
% SmallCapsFont = Fontin-SmallCaps.otf,
% BoldFont = Fontin-Bold.otf,
% ItalicFont = Fontin-Italic.otf
% ]
% {Fontin.otf}

%----------------------------------------------------------------------------------------
%	PACKAGES
%----------------------------------------------------------------------------------------
\usepackage{url}
\usepackage{parskip} 	

%other packages for formatting
\RequirePackage{color}
\RequirePackage{graphicx}
\usepackage[usenames,dvipsnames]{xcolor}
\usepackage[scale=0.9]{geometry}

%tabularx environment
\usepackage{tabularx}

%for lists within experience section
\usepackage{enumitem}

% centered version of 'X' col. type
\newcolumntype{C}{>{\centering\arraybackslash}X} 

%to prevent spillover of tabular into next pages
\usepackage{supertabular}
\usepackage{tabularx}
\newlength{\fullcollw}
\setlength{\fullcollw}{0.47\textwidth}

%custom \section
\usepackage{titlesec}				
\usepackage{multicol}
\usepackage{multirow}

%CV Sections inspired by: 
%http://stefano.italians.nl/archives/26
\titleformat{\section}{\Large\scshape\raggedright}{}{0em}{}[\titlerule]
\titlespacing{\section}{0pt}{10pt}{10pt}

%for publications
\usepackage[style=authoryear,sorting=ynt, maxbibnames=2]{biblatex}

%Setup hyperref package, and colours for links
\usepackage[unicode, draft=false]{hyperref}
\definecolor{linkcolour}{rgb}{0,0.2,0.6}
\hypersetup{colorlinks,breaklinks,urlcolor=linkcolour,linkcolor=linkcolour}
\addbibresource{citations.bib}
\setlength\bibitemsep{1em}

%for social icons
\usepackage{fontawesome5}

%debug page outer frames
%\usepackage{showframe}


% job listing environments
\newenvironment{jobshort}[2]
    {
    \begin{tabularx}{\linewidth}{@{}l X r@{}}
    \textbf{#1} & \hfill &  #2 \\[3.75pt]
    \end{tabularx}
    }
    {
    }

\newenvironment{joblong}[2]
    {
    \begin{tabularx}{\linewidth}{@{}l X r@{}}
    \textbf{#1} & \hfill &  #2 \\[3.75pt]
    \end{tabularx}
    \begin{minipage}[t]{\linewidth}
    \begin{itemize}[nosep,after=\strut, leftmargin=1em, itemsep=3pt,label=--]
    }
    {
    \end{itemize}
    \end{minipage}    
    }



%----------------------------------------------------------------------------------------
%	BEGIN DOCUMENT
%----------------------------------------------------------------------------------------
\begin{document}

% non-numbered pages
\pagestyle{empty} 

%----------------------------------------------------------------------------------------
%	TITLE
%----------------------------------------------------------------------------------------

% \begin{tabularx}{\linewidth}{ @{}X X@{} }
% \huge{Reed M. Milewicz}\vspace{2pt} & \hfill \emoji{incoming-envelope} rmmilewi@gmail.com \\
% \raisebox{-0.05\height}\faGithub\ rmmilewi \ | \
% \raisebox{-0.00\height}\faLinkedin\ reed-milewicz \ | \ \raisebox{-0.05\height}\faGlobe \ rmmilewi@gmail.com  & \hfill \emoji{calling} 205-612-0779
% \end{tabularx}

\begin{tabularx}{\linewidth}{@{} C @{}}
\Huge{Your Name} \\[7.5pt]
\href{https://github.com/username}{\raisebox{-0.05\height}\faGithub\ username} \ $|$ \ 
\href{https://www.linkedin.com/in/reed-milewicz-552a41b/}{\raisebox{-0.05\height}\faLinkedin\ reed-milewicz} \ $|$ \ 
\href{https://rmmilewi.github.io}{\raisebox{-0.05\height}\faGlobe \ rmmilewi.github.io} \ $|$ \ 
\href{mailto:rmmilewi@gmail.com}{\raisebox{-0.05\height}\faEnvelope \ rmmilewi@gmail.com} \ $|$ \ 
\href{tel:+2056120779}{\raisebox{-0.05\height}\faMobile \ +205.612.0779} \\
\end{tabularx}

%----------------------------------------------------------------------------------------
% EXPERIENCE SECTIONS
%----------------------------------------------------------------------------------------

%Interests/ Keywords/ Summary
\section{Summary}
Reed M. Milewicz Ph.D., is a computer scientist and software engineering researcher in the Department of Software Engineering and Research at Sandia National Laboratories. His current research focuses on developing better practices, processes, and tools to improve software development in the scientific domain. This is a course of research that straddles the line between systems and human factors, ranging from technologies such as compilers and formal verification tools to direct engagement with software teams through evidence-based software process improvement.

%Experience
\section{Work Experience}

\begin{joblong}{Senior Member of Technical Staff at Sandia National Laboratories}{Spring 2019 - Present}
\item Lead original research in software engineering and computing at the intersection of AI, HPC, and high-consequence systems,
\item Serve as Principal Investigator and project lead for multiple projects, responsible for defining research vision, managing multidisciplinary teams, and ensuring successful delivery of technical milestones and sponsor outcomes.
\item Develop and win competitive research proposals to DOE, DoD, and other federal agencies, securing funding to sustain and grow strategic R\&D initiatives.
\item Author and co-author peer-reviewed publications in top-tier conferences and journals, disseminating novel methods and findings to the academic and industrial research communities.
\item Build and maintain strong collaborations across national laboratories, universities, and industry, as well as cross-directorate partnerships within Sandia, to extend research impact and influence.
\end{joblong}

\begin{jobshort}{Postdoctoral Appointee at Sandia National Laboratories}{Fall 2016 - Spring 2019}
Performed software engineering research targeting scientific software development, and, as part of the Productivity and Sustainability Improvement Planning (PSIP) team within the Interoperable Design of Extreme-Scale Application Software (IDEAS) project, coordinating and conducting PSIP activities with partners across the Exascale Computing Project (ECP).
\end{jobshort}

\begin{jobshort}{Graduate Research Assistant}{Fall 2013 - Spring 2016}
Worked as part of the iProgress lab (PI: Dr. Peter Pirkelbauer) to develop innovative research while concurrently pursuing a doctoral degree. Was directly responsible for coordinating undergraduate research activities.
\end{jobshort}

\begin{jobshort}{Research Intern at Lawrence Livermore National Laboratory}{Summer 2013}
Developed critical software components for the ROSE compiler framework, in particular memory management around AST node deletion.
\end{jobshort}

\begin{jobshort}{Teaching Assistant and Student Instructor at University of Alabama at Birmingham}{2012-2013}
Was responsible for grading assignments, tutoring students, and teaching lab sessions. Taught an introductory course in programming using the Python language, with a focus on multimedia applications, where I was responsible for all lectures and labs.
\end{jobshort}
  
%Projects
\section{Projects}

\begin{tabularx}{\linewidth}{ @{}l r@{} }
\textbf{Some Project} & \hfill \href{https://some-link.com}{Link to Demo} \\[3.75pt]
\multicolumn{2}{@{}X@{}}{long long line of blah blah that will wrap when the table fills the column width long long line of blah blah that will wrap when the table fills the column width long long line of blah blah that will wrap when the table fills the column width long long line of blah blah that will wrap when the table fills the column width}  \\
\end{tabularx}

%----------------------------------------------------------------------------------------
%	EDUCATION
%----------------------------------------------------------------------------------------
\section{Education}
\begin{tabularx}{\linewidth}{@{}l X@{}}	
2030 - present & PhD (Subject) at \textbf{University} \hfill \normalsize (GPA: 4.0/4.0) \\

2023 - 2027 & Bachelor's Degree at \textbf{College} \hfill (GPA: 4.0/4.0) \\ 

2022 & Class 12th Some Board \hfill  (Grades) \\

2021 & Class 10th Some Board \hfill  (Grades) \\
\end{tabularx}

%----------------------------------------------------------------------------------------
%	PUBLICATIONS
%----------------------------------------------------------------------------------------
\section{Publications}
\begin{refsection}[citations.bib]
\nocite{*}
\printbibliography[heading=none]
\end{refsection}

%----------------------------------------------------------------------------------------
%	SKILLS
%----------------------------------------------------------------------------------------
\section{Skills}
\begin{tabularx}{\linewidth}{@{}l X@{}}
Some Skills &  \normalsize{This, That, Some of this and that etc.}\\
Some More Skills  &  \normalsize{Also some more of this, Some more that, And some of this and that etc.}\\  
\end{tabularx}

\vfill
\center{\footnotesize Last updated: \today}

\end{document}
