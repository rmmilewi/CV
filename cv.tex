%-----------------------------------------------------------------------------------------------------------------------------------------------%
%	The MIT License (MIT)
%
%	Copyright (c) 2021 Jitin Nair
%
%	Permission is hereby granted, free of charge, to any person obtaining a copy
%	of this software and associated documentation files (the "Software"), to deal
%	in the Software without restriction, including without limitation the rights
%	to use, copy, modify, merge, publish, distribute, sublicense, and/or sell
%	copies of the Software, and to permit persons to whom the Software is
%	furnished to do so, subject to the following conditions:
%	
%	THE SOFTWARE IS PROVIDED "AS IS", WITHOUT WARRANTY OF ANY KIND, EXPRESS OR
%	IMPLIED, INCLUDING BUT NOT LIMITED TO THE WARRANTIES OF MERCHANTABILITY,
%	FITNESS FOR A PARTICULAR PURPOSE AND NONINFRINGEMENT. IN NO EVENT SHALL THE
%	AUTHORS OR COPYRIGHT HOLDERS BE LIABLE FOR ANY CLAIM, DAMAGES OR OTHER
%	LIABILITY, WHETHER IN AN ACTION OF CONTRACT, TORT OR OTHERWISE, ARISING FROM,
%	OUT OF OR IN CONNECTION WITH THE SOFTWARE OR THE USE OR OTHER DEALINGS IN
%	THE SOFTWARE.
%	
%
%-----------------------------------------------------------------------------------------------------------------------------------------------%

%----------------------------------------------------------------------------------------
%	DOCUMENT DEFINITION
%----------------------------------------------------------------------------------------

% article class because we want to fully customize the page and not use a cv template
\documentclass[a4paper,12pt]{article}

%----------------------------------------------------------------------------------------
%	FONT
%----------------------------------------------------------------------------------------

% % fontspec allows you to use TTF/OTF fonts directly
% \usepackage{fontspec}
% \defaultfontfeatures{Ligatures=TeX}

% % modified for ShareLaTeX use
% \setmainfont[
% SmallCapsFont = Fontin-SmallCaps.otf,
% BoldFont = Fontin-Bold.otf,
% ItalicFont = Fontin-Italic.otf
% ]
% {Fontin.otf}

%----------------------------------------------------------------------------------------
%	PACKAGES
%----------------------------------------------------------------------------------------
\usepackage{url}
\usepackage{parskip} 	

%other packages for formatting
\RequirePackage{color}
\RequirePackage{graphicx}
\usepackage[usenames,dvipsnames]{xcolor}
\usepackage[scale=0.9]{geometry}

%tabularx environment
\usepackage{tabularx}

%for lists within experience section
\usepackage{enumitem}

% centered version of 'X' col. type
\newcolumntype{C}{>{\centering\arraybackslash}X} 

%to prevent spillover of tabular into next pages
\usepackage{supertabular}
\usepackage{tabularx}
\newlength{\fullcollw}
\setlength{\fullcollw}{0.47\textwidth}

%custom \section
\usepackage{titlesec}				
\usepackage{multicol}
\usepackage{multirow}

%CV Sections inspired by: 
%http://stefano.italians.nl/archives/26
\titleformat{\section}{\Large\scshape\raggedright}{}{0em}{}[\titlerule]
\titlespacing{\section}{0pt}{10pt}{10pt}

%for publications
\usepackage[style=authoryear,sorting=ynt, maxbibnames=2]{biblatex}

%Setup hyperref package, and colours for links
\usepackage[unicode, draft=false]{hyperref}
\definecolor{linkcolour}{rgb}{0,0.2,0.6}
\hypersetup{colorlinks,breaklinks,urlcolor=linkcolour,linkcolor=linkcolour}
\addbibresource{citations.bib}
\setlength\bibitemsep{1em}

%for social icons
\usepackage{fontawesome5}

%debug page outer frames
%\usepackage{showframe}


% job listing environments
\newenvironment{jobshort}[2]
    {
    \begin{tabularx}{\linewidth}{@{}l X r@{}}
    \textbf{#1} & \hfill &  #2 \\[3.75pt]
    \end{tabularx}
    }
    {
    }

\newenvironment{joblong}[2]
    {
    \begin{tabularx}{\linewidth}{@{}l X r@{}}
    \textbf{#1} & \hfill &  #2 \\[3.75pt]
    \end{tabularx}
    \begin{minipage}[t]{\linewidth}
    \begin{itemize}[nosep,after=\strut, leftmargin=1em, itemsep=3pt,label=--]
    }
    {
    \end{itemize}
    \end{minipage}    
    }



%----------------------------------------------------------------------------------------
%	BEGIN DOCUMENT
%----------------------------------------------------------------------------------------
\begin{document}

% non-numbered pages
\pagestyle{empty} 

%----------------------------------------------------------------------------------------
%	TITLE
%----------------------------------------------------------------------------------------

% \begin{tabularx}{\linewidth}{ @{}X X@{} }
% \huge{Reed M. Milewicz}\vspace{2pt} & \hfill \emoji{incoming-envelope} rmmilewi@gmail.com \\
% \raisebox{-0.05\height}\faGithub\ rmmilewi \ | \
% \raisebox{-0.00\height}\faLinkedin\ reed-milewicz \ | \ \raisebox{-0.05\height}\faGlobe \ rmmilewi@gmail.com  & \hfill \emoji{calling} 205-612-0779
% \end{tabularx}

\begin{tabularx}{\linewidth}{@{} C @{}}
\Huge{Reed M. Milewicz} \\[7.5pt]
\href{https://github.com/username}{\raisebox{-0.05\height}\faGithub\ username} \ $|$ \ 
\href{https://www.linkedin.com/in/reed-milewicz-552a41b/}{\raisebox{-0.05\height}\faLinkedin\ reed-milewicz} \ $|$ \ 
\href{https://rmmilewi.github.io}{\raisebox{-0.05\height}\faGlobe \ rmmilewi.github.io} \ $|$ \ 
\href{mailto:rmmilewi@gmail.com}{\raisebox{-0.05\height}\faEnvelope \ rmmilewi@gmail.com} \ $|$ \ 
\href{tel:+2056120779}{\raisebox{-0.05\height}\faMobile \ +205.612.0779} \\
\end{tabularx}

%----------------------------------------------------------------------------------------
% EXPERIENCE SECTIONS
%----------------------------------------------------------------------------------------

%Interests/ Keywords/ Summary
\section{Summary}
Reed M. Milewicz Ph.D., is a computer scientist and software engineering researcher in the Department of Software Engineering and Research at Sandia National Laboratories. His current research focuses on developing better practices, processes, and tools to improve software development in the scientific domain. This is a course of research that straddles the line between systems and human factors, ranging from technologies such as compilers and formal verification tools to direct engagement with software teams through evidence-based software process improvement.

%Experience
\section{Work Experience}

\begin{joblong}{Senior Member of Technical Staff at Sandia National Laboratories}{Spring 2019 - Present}
\item Lead original research in software engineering and computing at the intersection of AI, HPC, and high-consequence systems,
\item Serve as Principal Investigator and project lead for multiple projects, responsible for defining research vision, managing multidisciplinary teams, and ensuring successful delivery of technical milestones and sponsor outcomes.
\item Develop and win competitive research proposals to DOE, DoD, and other federal agencies, securing funding to sustain and grow strategic R\&D initiatives.
\item Author and co-author peer-reviewed publications in top-tier conferences and journals, disseminating novel methods and findings to the academic and industrial research communities.
\item Build and maintain strong collaborations across national laboratories, universities, and industry, as well as cross-directorate partnerships within Sandia, to extend research impact and influence.
\end{joblong}

\begin{jobshort}{Postdoctoral Appointee at Sandia National Laboratories}{Fall 2016 - Spring 2019}
Performed software engineering research targeting scientific software development, and, as part of the Productivity and Sustainability Improvement Planning (PSIP) team within the Interoperable Design of Extreme-Scale Application Software (IDEAS) project, coordinating and conducting PSIP activities with partners across the Exascale Computing Project (ECP).
\end{jobshort}

\begin{jobshort}{Graduate Research Assistant}{Fall 2013 - Spring 2016}
Worked as part of the iProgress lab (PI: Dr. Peter Pirkelbauer) to develop innovative research while concurrently pursuing a doctoral degree. Was directly responsible for coordinating undergraduate research activities.
\end{jobshort}

\begin{jobshort}{Research Intern at Lawrence Livermore National Laboratory}{Summer 2013}
Developed critical software components for the ROSE compiler framework, in particular memory management around AST node deletion.
\end{jobshort}

\begin{jobshort}{Teaching Assistant and Student Instructor at University of Alabama at Birmingham}{2012-2013}
Was responsible for grading assignments, tutoring students, and teaching lab sessions. Taught an introductory course in programming using the Python language, with a focus on multimedia applications, where I was responsible for all lectures and labs.
\end{jobshort}
  
%Projects
\section{Projects}

\begin{tabularx}{\linewidth}{ @{}l r@{} }
\textbf{Some Project} & \hfill \href{https://some-link.com}{Link to Demo} \\[3.75pt]
\multicolumn{2}{@{}X@{}}{long long line of blah blah that will wrap when the table fills the column width long long line of blah blah that will wrap when the table fills the column width long long line of blah blah that will wrap when the table fills the column width long long line of blah blah that will wrap when the table fills the column width}  \\
\end{tabularx}

%----------------------------------------------------------------------------------------
%	EDUCATION
%----------------------------------------------------------------------------------------
\section{Education}
\begin{tabularx}{\linewidth}{@{}l X@{}}	
Fall 2013--Summer 2016&PhD in Computer and Information Sciences, University of Alabama at Birmingham.\\[5pt]
&\textbf{Thesis Title}: Improving the Scalability of Directed Model Checking of Concurrent Java Code though Hybrid and Distributed Analysis.\\\\
&\textbf{Supervisor}: Dr. Peter Pirkelbauer.\\\\
&\textbf{Interesting aside}: At the time of my graduation, I held the unusual distinction of attaining both the highest scores on the qualifying exams and the fastest completion rate of my PhD in the history of the department. \\\\
Fall 2011--Spring 2013&MS in Computer and Information Sciences, University of Alabama at Birmingham.\\
Fall 2007--Spring 2011&BS in Computer Science with minors in Mathematics and Classics, Birmingham-Southern College.\\
\end{tabularx}


%----------------------------------------------------------------------------------------
%	PUBLICATIONS
%----------------------------------------------------------------------------------------
%\bibliographystyle{plain}
%\nobibliography{citations.bib}
\section{Recent Publications}
\begin{small}
%\begin{tabular}{L!{\VRule}R}
\begin{tabularx}{\linewidth}{@{}l X@{}}	
2025& Feng, Zixuan, \textbf{Reed Milewicz}, Emerson Murphy-Hill, Tyler Menezes, Alexander Serebrenik, Igor Steinmacher, and Anita Sarma. "Charting Uncertain Waters: A Socio-Technical Framework for Navigating GenAI's Impact on Open Source Communities. (Under Review)" arXiv preprint arXiv:2508.04921 (2025).\\[5 pt]
2025& Feng, Zixuan, Igor Steinmacher, Marco Gerosa, Tyler Menezes, Alexander Serebrenik, \textbf{Reed Milewicz}, and Anita Sarma. "The multifaceted nature of mentoring in oss: strategies, qualities, and ideal outcomes." In 2025 IEEE/ACM 18th International Conference on Cooperative and Human Aspects of Software Engineering (CHASE), pp. 203-214. IEEE, 2025.\\[5 pt]
2025& Thakur, Addi Malviya, \textbf{Reed Milewicz}, Mahmoud Jahanshahi, Lavínia Paganini, Bogdan Vasilescu, and Audris Mockus. "Scientific Open-Source Software Is Less Likely to Become Abandoned Than One Might Think! Lessons from Curating a Catalog of Maintained Scientific Software." Proceedings of the ACM on Software Engineering 2, no. FSE (2025): 2216-2239.\\[5 pt]
2024& Gilbertson, Christian, \textbf{Reed Milewicz}, Eric Berquist, Aaron Brundage, John Engelmann, Brian Evans, Nicholas Francis et al. "Towards Long-Term Scientific Model Sustainment at Sandia National Laboratories." In Proceedings of the 39th IEEE/ACM International Conference on Automated Software Engineering, pp. 2142-2147. 2024.\\[5 pt]
2024& Jacobs, Mariska, \textbf{Reed Milewicz}, and Alexander Serebrenik. "Mentorship of women in oss projects: A cross-disciplinary, integrative review." Equity, Diversity, and Inclusion in Software Engineering: Best Practices and Insights (2024): 337-364.\\[5 pt]
2024& Gilbertson, Christian, Miranda Mundt, Joshua Teves, Simone Toribio, and \textbf{Reed Milewicz}. "Towards Evidence-Based Software Quality Practices for Reproducibility: Practices and Aligned Software Qualities." In Proceedings of the 2nd ACM Conference on Reproducibility and Replicability, pp. 52-63. 2024.
2023&\textbf{Reed Milewicz}, Jonathan Bisila, Miranda Mundt, Sylvain Bernard, Michael Robert Buche, Jason M. Gates, Samuel Andrew Grayson, Evan Harvey, Alexander Jaeger, Kirk Timothy Landin, Mitchell Negus, Bethany L. Nicholson. DevOps Pragmatic Practices and Potential Perils in Scientific Software Development. Proceedings of the 8th International Congress on Information and Communication Technology. 2023. Springer. 20 pages.\\[5 pt]
2023&Mundt, M. R., Beattie, K., Bisila, J., Ferenbaugh, C. R., Godoy, W. F., Gupta, R.,Guyer, J.E., Kiran, M., Malviya-Thakur, A., \textbf{Milewicz, R.}, Sims, B.H., Sochat, V., Teves, J.B. (2023). For the Public Good: Connecting, Retaining, and Recognizing Current and Future RSEs at National Organizations. Computing in Science and Engineering. 2023.\\[5 pt]
2023&Miranda Mundt, Jonathan Bisila, \textbf{Reed Milewicz}, Joshua Teves, Michael Buche, Jonathan Compton, Jason M. Gates, Kirk Landin, Gerald Lofstead. 2023. Challenges and Strategies for Testing Automation Practices at Sandia National Laboratories. 1st Annual Conference of the United States Association of Research Software Engineers (US-RSE’23). 10 pages.\\[5 pt]
2023&\textbf{Reed Milewicz}, Jonathan Bisila, Miranda Mundt, and Joshua Teves. Seeking Enlightenment: An Experience Report on Incorporating Evidence-Based Practice Techniques in a Research Software Engineering Team.  1st Annual Conference of the United States Association of Research Software Engineers (US-RSE’23). 10 pages.\\[5 pt]
2023&Grayson, Samuel, Darko Marinov, Daniel S. Katz, and \textbf{Reed Milewicz}. "Automatic Reproduction of Workflows in the Snakemake Workflow Catalog and nf-core Registries." In Proceedings of the 2023 ACM Conference on Reproducibility and Replicability, pp. 74-84. 2023.\\[5 pt]
\end{tabular}
\end{small}

%\begin{refsection}[citations.bib]
%\nocite{*}
%\printbibliography[heading=none]
%\end{refsection}

%----------------------------------------------------------------------------------------
%	SKILLS
%----------------------------------------------------------------------------------------
\section{Skills}
\begin{tabularx}{\linewidth}{@{}l X@{}}
Some Skills &  \normalsize{This, That, Some of this and that etc.}\\
Some More Skills  &  \normalsize{Also some more of this, Some more that, And some of this and that etc.}\\  
\end{tabularx}

\vfill
\center{\footnotesize Last updated: \today}

\end{document}
